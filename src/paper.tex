The control of sophisticated prosthetic limbs generally relies on skin-surface electrodes.
Lancashire et al. take advantage of the transcutaneous nature of an ITAP to implant electrodes on epymisia (the surface of muscles), and then use the prosthesis as a portal for a wired connection to the outside of the body.
That is, the purpose of the ITAP is to allow wires to go from an interface outside of the body directly into the electrodes that are placed on the muscles.
In short, the position of the electrodes improves the quality of the EMG readings, reduces noise, and removes user error from continuous replacement.
The ITAP provides a simple way path the wiring and to seal the channel between the outside and inside of the body.

The study aims to validate this approach by extending the timescale from previous studies: From 12 to 19 weeks and increase the number of participant animals, n = 6.
It also explores the effects of targetted muscle reinnervation (TMR) for one animal over 12 weeks.
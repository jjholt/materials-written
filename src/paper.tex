The control of sophisticated prosthetic limbs generally relies on skin-surface electrodes.
Lancashire et al. take advantage of the transcutaneous nature of an ITAP to implant electrodes on epymisia (the surface of muscles), and then use the prosthesis as a portal for a wired connection to the outside of the body.
The position of these electrodes improve the quality of the EMG readings, reduce noise, and don't require the user to replace them periodically.

The study aims to validate this approach by extending the timescale (From 12 to 19 weeks) and number of participant animals (n = 6) from previous studies.
It also explores the effects of targetted muscle reinnervation (TMR) for one animal over 12 weeks.
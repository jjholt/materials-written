\subsection{Morphology}
Cutaneous integration was subpar for one animal for whom the ITAP was placed proximally on the tibia.
Despite no signs of infection, there was necrosis of the skin after 8 weeks.
The other animals with more distal prostheses displayed better outcomes.
They showed good dermal integration into 90\% of pores with good vascularisation of 71 blood vessels per \si{\milli\metre\squared}.
In places of poor integration, both downgrowth and the formation of sinuses was observed, which allowed the collection of debris.

For the sheep who underwent TMR, some muscular atrophy was noted on the muscle whose nerves were transected. While atrophy is expected in TMR, often the muscle is already imobile, but that was not the case in this study.

\subsection{EMG}
Epimysial EMG readings showed a reduction in noise as compared to skin readings. Typical signal-to-noise ratios (SNR) ranged from \SIrange{10}{25}{\deci\bel} week-to-week and \SI{19.6 (74)}{\deci\bel} at 19 weeks.
Skin surface was \SI{6.65(763)}{\deci\bel}, however they were statistically different.

\subsection{Others}
Crosstalk is the amount of signal caused by the activation of nearby muscles which were largest when stimulating the peroneal nerve muscles and smallest when stimulating the gastrocnemius.
These were between one and three orders of magnitude smaller than the targeted muscle signal.

Mean electrode impedance went from \SI{1.3}{\kilo\ohm} on implantantion to \SI{2.2}{\kilo\ohm} in-vivo, to \SI{3.1}{\kilo\ohm} at explantation and placement in saline.

For the animal who underwent TMR, after 6 weeks weight bearing is recovered, but it is only possible to distinguish the EMG signal from crosstalk at 10 weeks.  
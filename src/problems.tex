Stress shielding is a problem presented by most bone-anchored prosthesis, wherein the presence of the prosthesis results in non-physiological stresses, osteopoenia and ultimately aseptic loosening.
There is research in producing porous implants \cite{DhandapaniRamya2020Amob}, modifying stiffness \cite{AhmedK.2020EVoa}, subject-specific wear and a lot more, to reduce this particular negative outcome.
It's worth noting aseptic loosening is large problem with several sources, such as poor osseointegration, the above-mentioned bone density loss, debris causing osteolysis, periprosthetic failures, and a lot more.

Given this setup, a prosthetic limb would have some delay due to the EMG recording, then more delay for the motor signals. This process is likely to take much longer than secondary cues users tend to rely on \cite{schofield_applications_2014}, e.g., localised pressure on the residual limb or a particular sound it makes.
That is, there is likely to be an unintuitive gap between usage and feedback.

The market for transcutaneous bone-anchored devices is minuscule and it is not broadly offered to patients by national health systems, though this research is impactful for the small cohort that does exist; and worse outcomes because of reduced surgical practice \cite{MaruthappuMahiben2015TIoV}.

Whether vibrations of the relevant amplitude and frequency are conducive to aseptic loosening (i.e. worsening outcomes), or whether feedback on bone is ``better'' than on the skin are unanswered questions.
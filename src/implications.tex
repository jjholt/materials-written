\subsection{Patients}
Generally, prostheses users rely on visual feedback and secondary cues, such as sound or changes in pressure on the residual limb to recover the lost sensory information \cite{schofield_applications_2014}.
Sensory feedback, particularly through the use of vibrations, is associated with better object recognition, manipulation and grip force control \cite{schiefer_artificial_2018, rosenbaum-chou_development_2016}.
Traditional stump prostheses must stimulate the surface of the skin to produce some sensory feedback, but bone-anchored devices enable the provision of feedback directly onto bone \cite{OrgelMarcus2021Oito}, taking advantage of bones' ability to perceive vibrations.

Whether vibrations of the relevant amplitude and frequency are conducive to aseptic loosening, or feedback is better on the bone or skin are unanswered questions.

\subsection{Clinical use}

The most important takeaway is that this study provides further support to the use of transcutaneous bone-anchored devices as a tool to control sophisticated prosthetic limbs.
\subsection{Patients}
Lancashire et al. explore the possibility of concurrently implanting an ITAP and electrodes for myoelectric control of prosthetic limbs.
This paper puts us in a position to explore receiving information from the prosthetic limb and closing the loop that restores some feeling and proprioception to amputees.

Prostheses users tend to rely on visual feedback and secondary cues, such as sound or changes in pressure on the residual limb, for sensory information \cite{schofield_applications_2014}.
Providing sensory feedback results in better object recognition, manipulation and grip force control \cite{schiefer_artificial_2018, rosenbaum-chou_development_2016}.
In order to produce this feedback, traditional stump prostheses require stimulation of the surface of the skin, but bone-anchored devices enable the provision of feedback directly onto bone, taking advantage of bones' ability to perceive vibrations through osseoperception \cite{OrgelMarcus2021Oito}.

The development of ITAP and a feedback system have, among other advantages \cite{Hagberg2008}, a reduction skin in abrasions, more stable prosthetic limb attachment, and ultimately another avenue to explore ``restoring'' limb function.
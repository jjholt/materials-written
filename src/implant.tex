Not only does amputation result in limb loss, but also loss in the perception enabled by the limb --- removing a fundamental way we gather information about the world around us.
Following an amputation, stump-socket prostheses are used to restore function by attaching and securing the device onto the residual limb.
This results in soft tissue being responsible for non-physiological weight bearing, causing skin abrasions \cite{MeulenbeltHenkEJ2011SPot}, osteopoenia and ultimately patients opting not to use their prostheses.

Intraosseous Transcutaneous Amputation Prostheses (ITAP) \cite{kang_osseocutaneous_2010} are an alternative attachment method which bypasses the traditional load-bearing role of the residuum.
They achieve this by anchoring the device into bone and enabling stresses to be transmitted from bone through prosthesis to prosthetic limb --- allowing the physiological loads necessary to maintain bone health.
Furthermore, the presence of a porous flange promotes cutaneous integration akin to a deer's antlers \cite{PendegrassC.J.2006Natb}.
The study used a cylindrical flange on the ITAP, which disagreed with the traditional flange shape \cite{kang_osseocutaneous_2010}.
They speculated that the traditional dome shape would ultimately improve cutaneous integration and reduce some the observed skin damage.
Furthermore, allowing the wires to flow from the flange directly into soft tissue enables a simpler surgery that doesn't require making holes on the cortex of the bone.
Every path to the outside of the body requires a hermetic seal (e.g. ceramic-to-metal feedthrough), however for implantation in animal, the authors used medical grade silicone.

For the TMR case, the use of SNR as a measuring tool does not convey subtle changes, such as a reduction in EMG frequency.
There is a broader desire to use TMR because of its role in restoring proprioception and some sensory feedback.
Additionally, there was the inability to discern the effect of reinnervation and the surgery on weight bearing recovery.
The authors suggest a control condition which inhibits particular reinnervation and would enable that differentiation.
Further details are not discussed in this paper because we do not go into detail about the exact nerves being transected.

The use of bipolar electrode arrays limited recordings to one location.
Lancashire et al. suggested the use of multielectrode arrays to gather more information about moving intention.
That is, giving the ability to have considerably more information for control of the eventual prosthetic limb.
Furthermore, tripolar electrodes and a reduction in interelectrode distance could be used to reduce crosstalk.

The authors final considerations are that the increase in electrode impedance in-vivo is in line with previous studies and the approach discussed is not limited to EMG and could equally be used for neural recordings.
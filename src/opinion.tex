The main alternatives to the devices described by Lancashire et al. are 1) skin-surface electrodes, 2) stump prostheses, and 3) Brånemark prostheses (and similar bone-anchored prostheses) \cite{LiYan2021Tbap}.
Alternatives 1 and 2 are well-established and motivation for their use has been described throughout this paper, therefore, focus is given to alternative 3 and why ITAP is a better implant.

Brånemark prostheses provide no means of closing the skin barrier. That is, there is a path from the outside to the inside of the body that leads to soft-tissue infections, chronic discharge and other pathologies \cite{kang_osseocutaneous_2010}.
The interface between skin and the implant requires cutaneous integration to reduce the soft tissue mobility and to create that needed barrier.
This is achieved by the presence of a porous flange to which soft tissue is adhered.

The implantantion of a Brånemark prosthesis requires two separate surgeries, resulting in longer rehabilitation. Furthermore, the second implanted component is prone to bending, fracture and loosening.
ITAP consists of only one surgical step of a titanium 6--4 device --- though that presents a catastrophic failure were it to fracture \cite{kang_osseocutaneous_2010}. 